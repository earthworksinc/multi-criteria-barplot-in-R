% Options for packages loaded elsewhere
\PassOptionsToPackage{unicode}{hyperref}
\PassOptionsToPackage{hyphens}{url}
%
\documentclass[
]{article}
\usepackage{lmodern}
\usepackage{amssymb,amsmath}
\usepackage{ifxetex,ifluatex}
\ifnum 0\ifxetex 1\fi\ifluatex 1\fi=0 % if pdftex
  \usepackage[T1]{fontenc}
  \usepackage[utf8]{inputenc}
  \usepackage{textcomp} % provide euro and other symbols
\else % if luatex or xetex
  \usepackage{unicode-math}
  \defaultfontfeatures{Scale=MatchLowercase}
  \defaultfontfeatures[\rmfamily]{Ligatures=TeX,Scale=1}
\fi
% Use upquote if available, for straight quotes in verbatim environments
\IfFileExists{upquote.sty}{\usepackage{upquote}}{}
\IfFileExists{microtype.sty}{% use microtype if available
  \usepackage[]{microtype}
  \UseMicrotypeSet[protrusion]{basicmath} % disable protrusion for tt fonts
}{}
\makeatletter
\@ifundefined{KOMAClassName}{% if non-KOMA class
  \IfFileExists{parskip.sty}{%
    \usepackage{parskip}
  }{% else
    \setlength{\parindent}{0pt}
    \setlength{\parskip}{6pt plus 2pt minus 1pt}}
}{% if KOMA class
  \KOMAoptions{parskip=half}}
\makeatother
\usepackage{xcolor}
\IfFileExists{xurl.sty}{\usepackage{xurl}}{} % add URL line breaks if available
\IfFileExists{bookmark.sty}{\usepackage{bookmark}}{\usepackage{hyperref}}
\hypersetup{
  pdftitle={geom\_bar.R},
  pdfauthor={shayan},
  hidelinks,
  pdfcreator={LaTeX via pandoc}}
\urlstyle{same} % disable monospaced font for URLs
\usepackage[margin=1in]{geometry}
\usepackage{color}
\usepackage{fancyvrb}
\newcommand{\VerbBar}{|}
\newcommand{\VERB}{\Verb[commandchars=\\\{\}]}
\DefineVerbatimEnvironment{Highlighting}{Verbatim}{commandchars=\\\{\}}
% Add ',fontsize=\small' for more characters per line
\usepackage{framed}
\definecolor{shadecolor}{RGB}{248,248,248}
\newenvironment{Shaded}{\begin{snugshade}}{\end{snugshade}}
\newcommand{\AlertTok}[1]{\textcolor[rgb]{0.94,0.16,0.16}{#1}}
\newcommand{\AnnotationTok}[1]{\textcolor[rgb]{0.56,0.35,0.01}{\textbf{\textit{#1}}}}
\newcommand{\AttributeTok}[1]{\textcolor[rgb]{0.77,0.63,0.00}{#1}}
\newcommand{\BaseNTok}[1]{\textcolor[rgb]{0.00,0.00,0.81}{#1}}
\newcommand{\BuiltInTok}[1]{#1}
\newcommand{\CharTok}[1]{\textcolor[rgb]{0.31,0.60,0.02}{#1}}
\newcommand{\CommentTok}[1]{\textcolor[rgb]{0.56,0.35,0.01}{\textit{#1}}}
\newcommand{\CommentVarTok}[1]{\textcolor[rgb]{0.56,0.35,0.01}{\textbf{\textit{#1}}}}
\newcommand{\ConstantTok}[1]{\textcolor[rgb]{0.00,0.00,0.00}{#1}}
\newcommand{\ControlFlowTok}[1]{\textcolor[rgb]{0.13,0.29,0.53}{\textbf{#1}}}
\newcommand{\DataTypeTok}[1]{\textcolor[rgb]{0.13,0.29,0.53}{#1}}
\newcommand{\DecValTok}[1]{\textcolor[rgb]{0.00,0.00,0.81}{#1}}
\newcommand{\DocumentationTok}[1]{\textcolor[rgb]{0.56,0.35,0.01}{\textbf{\textit{#1}}}}
\newcommand{\ErrorTok}[1]{\textcolor[rgb]{0.64,0.00,0.00}{\textbf{#1}}}
\newcommand{\ExtensionTok}[1]{#1}
\newcommand{\FloatTok}[1]{\textcolor[rgb]{0.00,0.00,0.81}{#1}}
\newcommand{\FunctionTok}[1]{\textcolor[rgb]{0.00,0.00,0.00}{#1}}
\newcommand{\ImportTok}[1]{#1}
\newcommand{\InformationTok}[1]{\textcolor[rgb]{0.56,0.35,0.01}{\textbf{\textit{#1}}}}
\newcommand{\KeywordTok}[1]{\textcolor[rgb]{0.13,0.29,0.53}{\textbf{#1}}}
\newcommand{\NormalTok}[1]{#1}
\newcommand{\OperatorTok}[1]{\textcolor[rgb]{0.81,0.36,0.00}{\textbf{#1}}}
\newcommand{\OtherTok}[1]{\textcolor[rgb]{0.56,0.35,0.01}{#1}}
\newcommand{\PreprocessorTok}[1]{\textcolor[rgb]{0.56,0.35,0.01}{\textit{#1}}}
\newcommand{\RegionMarkerTok}[1]{#1}
\newcommand{\SpecialCharTok}[1]{\textcolor[rgb]{0.00,0.00,0.00}{#1}}
\newcommand{\SpecialStringTok}[1]{\textcolor[rgb]{0.31,0.60,0.02}{#1}}
\newcommand{\StringTok}[1]{\textcolor[rgb]{0.31,0.60,0.02}{#1}}
\newcommand{\VariableTok}[1]{\textcolor[rgb]{0.00,0.00,0.00}{#1}}
\newcommand{\VerbatimStringTok}[1]{\textcolor[rgb]{0.31,0.60,0.02}{#1}}
\newcommand{\WarningTok}[1]{\textcolor[rgb]{0.56,0.35,0.01}{\textbf{\textit{#1}}}}
\usepackage{graphicx,grffile}
\makeatletter
\def\maxwidth{\ifdim\Gin@nat@width>\linewidth\linewidth\else\Gin@nat@width\fi}
\def\maxheight{\ifdim\Gin@nat@height>\textheight\textheight\else\Gin@nat@height\fi}
\makeatother
% Scale images if necessary, so that they will not overflow the page
% margins by default, and it is still possible to overwrite the defaults
% using explicit options in \includegraphics[width, height, ...]{}
\setkeys{Gin}{width=\maxwidth,height=\maxheight,keepaspectratio}
% Set default figure placement to htbp
\makeatletter
\def\fps@figure{htbp}
\makeatother
\setlength{\emergencystretch}{3em} % prevent overfull lines
\providecommand{\tightlist}{%
  \setlength{\itemsep}{0pt}\setlength{\parskip}{0pt}}
\setcounter{secnumdepth}{-\maxdimen} % remove section numbering

\title{geom\_bar.R}
\author{shayan}
\date{2020-12-25}

\begin{document}
\maketitle

\begin{Shaded}
\begin{Highlighting}[]
\CommentTok{#libraries required for basic bar graph}
\KeywordTok{library}\NormalTok{(ggplot2)}
\KeywordTok{library}\NormalTok{(devtools)}
\end{Highlighting}
\end{Shaded}

\begin{verbatim}
## Loading required package: usethis
\end{verbatim}

\begin{Shaded}
\begin{Highlighting}[]
\KeywordTok{library}\NormalTok{(dplyr)}
\end{Highlighting}
\end{Shaded}

\begin{verbatim}
## 
## Attaching package: 'dplyr'
\end{verbatim}

\begin{verbatim}
## The following objects are masked from 'package:stats':
## 
##     filter, lag
\end{verbatim}

\begin{verbatim}
## The following objects are masked from 'package:base':
## 
##     intersect, setdiff, setequal, union
\end{verbatim}

\begin{Shaded}
\begin{Highlighting}[]
\CommentTok{#read csv function for reading the file from internet, usb or local drive}
\NormalTok{guidelines <-}\StringTok{ }\KeywordTok{read.csv}\NormalTok{(}\StringTok{"/Volumes/Lexar/Lexar/EarthWorks/Wastewater Guidelines_columned.csv"}\NormalTok{, }\DataTypeTok{header=}\OtherTok{TRUE}\NormalTok{)}

\CommentTok{#this the variable for the data file I am using. Typing the name will display the cintents of the file}
\NormalTok{guidelines}
\end{Highlighting}
\end{Shaded}

\begin{verbatim}
##   Parameter       X  Range Influent Effluent Guideline    Legend
## 1       BOD gBOD/m3 medium    310.0     12.0        30  Influent
## 2       COD gCOD/m3 medium    750.0     57.0        25  Effluent
## 3        SS  gSS/m3 medium    285.0     14.0        30 Guideline
## 4         P   gP/m3 medium     11.5      2.2         1          
## 5       TKN   gN/m3 medium     62.5      9.0        11
\end{verbatim}

\begin{Shaded}
\begin{Highlighting}[]
\CommentTok{#a good practive to summarize your data. I learned this from video tutorials of data science gurus}
\KeywordTok{summary}\NormalTok{(guidelines)}
\end{Highlighting}
\end{Shaded}

\begin{verbatim}
##   Parameter              X                Range              Influent    
##  Length:5           Length:5           Length:5           Min.   : 11.5  
##  Class :character   Class :character   Class :character   1st Qu.: 62.5  
##  Mode  :character   Mode  :character   Mode  :character   Median :285.0  
##                                                           Mean   :283.8  
##                                                           3rd Qu.:310.0  
##                                                           Max.   :750.0  
##     Effluent       Guideline       Legend         
##  Min.   : 2.20   Min.   : 1.0   Length:5          
##  1st Qu.: 9.00   1st Qu.:11.0   Class :character  
##  Median :12.00   Median :25.0   Mode  :character  
##  Mean   :18.84   Mean   :19.4                     
##  3rd Qu.:14.00   3rd Qu.:30.0                     
##  Max.   :57.00   Max.   :30.0
\end{verbatim}

\begin{Shaded}
\begin{Highlighting}[]
\CommentTok{#ggplot funtion (your_data, aes(x parameter, y parameter))}
\KeywordTok{ggplot}\NormalTok{(guidelines, }\KeywordTok{aes}\NormalTok{(Influent, Parameter))}\OperatorTok{+}
\StringTok{  }
\StringTok{  }\CommentTok{#geom_bar(stat, fill and width of the bar) --- [this is the bar plot based on the x and y defined above]}
\StringTok{  }\KeywordTok{geom_bar}\NormalTok{(}\DataTypeTok{stat=}\StringTok{"identity"}\NormalTok{, }\DataTypeTok{fill=}\StringTok{"Grey60"}\NormalTok{, }\DataTypeTok{width =} \FloatTok{0.25}\NormalTok{)}\OperatorTok{+}
\StringTok{  }
\StringTok{  }\CommentTok{#geom_bar(your parameter of choice, stat, fill and width of the bar)}
\StringTok{  }\KeywordTok{geom_bar}\NormalTok{(}\KeywordTok{aes}\NormalTok{(Effluent),}\DataTypeTok{stat=}\StringTok{"identity"}\NormalTok{, }\DataTypeTok{fill=}\StringTok{"blue"}\NormalTok{, }\DataTypeTok{width =} \FloatTok{0.25}\NormalTok{)}\OperatorTok{+}
\StringTok{  }
\StringTok{  }\CommentTok{#geom_errorbar(your parameter of choice(x, ymin, ymax), width, color and size) --- [very cool! adds a line at the end of your bar] }
\StringTok{  }\KeywordTok{geom_errorbar}\NormalTok{(}\KeywordTok{aes}\NormalTok{(}\DataTypeTok{x=}\NormalTok{Effluent, }\DataTypeTok{ymin=}\NormalTok{Parameter, }\DataTypeTok{ymax=}\NormalTok{Parameter),}\DataTypeTok{width=}\FloatTok{1.5}\NormalTok{, }\DataTypeTok{color=}\StringTok{"green"}\NormalTok{,}\DataTypeTok{size=}\DecValTok{15}\NormalTok{)}\OperatorTok{+}
\StringTok{  }
\StringTok{  }\CommentTok{#geom_point(aesthetics, color, fill, size, shape\{number 25 for the upside down triangle\}, nudge to find the spot)}
\StringTok{  }\KeywordTok{geom_point}\NormalTok{(}\KeywordTok{aes}\NormalTok{(Guideline),}\DataTypeTok{color=}\StringTok{"orange"}\NormalTok{, }\DataTypeTok{fill=}\StringTok{"orange"}\NormalTok{, }\DataTypeTok{size=}\FloatTok{1.5}\NormalTok{, }\DataTypeTok{shape=}\DecValTok{25}\NormalTok{, }\DataTypeTok{position =} \KeywordTok{position_nudge}\NormalTok{(}\DataTypeTok{y=}\OperatorTok{-}\FloatTok{0.083}\NormalTok{))}\OperatorTok{+}
\StringTok{  }
\StringTok{  }\CommentTok{#geom_text to get the values for the parameters on the panel}
\StringTok{  }\KeywordTok{geom_text}\NormalTok{(}\DataTypeTok{data=}\NormalTok{guidelines, }\KeywordTok{aes}\NormalTok{(}\DataTypeTok{label=}\NormalTok{Influent, }\DataTypeTok{x=}\NormalTok{Influent), }\DataTypeTok{stat=}\StringTok{"identity"}\NormalTok{, }\DataTypeTok{position=}\KeywordTok{position_dodge}\NormalTok{(}\DataTypeTok{width=}\FloatTok{0.9}\NormalTok{),}\DataTypeTok{size=}\DecValTok{4}\NormalTok{,}\DataTypeTok{hjust=}\OperatorTok{-}\FloatTok{0.2}\NormalTok{,}\DataTypeTok{color=}\StringTok{"Grey60"}\NormalTok{,}\DataTypeTok{fontface=}\StringTok{'bold'}\NormalTok{)}\OperatorTok{+}
\StringTok{  }\KeywordTok{geom_text}\NormalTok{(}\DataTypeTok{data=}\NormalTok{guidelines, }\KeywordTok{aes}\NormalTok{(}\DataTypeTok{label=}\NormalTok{Effluent, }\DataTypeTok{x=}\NormalTok{Effluent), }\DataTypeTok{stat=}\StringTok{"identity"}\NormalTok{, }\DataTypeTok{position=}\KeywordTok{position_dodge}\NormalTok{(}\DataTypeTok{width=}\FloatTok{0.9}\NormalTok{), }\DataTypeTok{size=}\FloatTok{3.5}\NormalTok{, }\DataTypeTok{hjust=}\FloatTok{0.6}\NormalTok{,}\DataTypeTok{vjust=}\OperatorTok{-}\FloatTok{3.}\NormalTok{, }\DataTypeTok{color=}\StringTok{"blue"}\NormalTok{)}\OperatorTok{+}
\StringTok{  }
\StringTok{  }\CommentTok{#I used geom_label here to make the numbers stand out from the grid background \{parse for bold fonts did not return bold fonts???\}}
\StringTok{  }\KeywordTok{geom_label}\NormalTok{(}\DataTypeTok{data=}\NormalTok{guidelines, }\KeywordTok{aes}\NormalTok{(}\DataTypeTok{label=}\NormalTok{Guideline, }\DataTypeTok{x=}\NormalTok{Guideline), }\DataTypeTok{stat=}\StringTok{"identity"}\NormalTok{, }\DataTypeTok{position=}\KeywordTok{position_dodge}\NormalTok{(}\DataTypeTok{width=}\FloatTok{0.9}\NormalTok{), }\DataTypeTok{parse =} \OtherTok{TRUE}\NormalTok{, }\DataTypeTok{size=}\FloatTok{3.6}\NormalTok{, }\DataTypeTok{hjust=}\FloatTok{0.4}\NormalTok{,}\DataTypeTok{vjust=}\FloatTok{2.5}\NormalTok{, }\DataTypeTok{color=}\StringTok{"orange"}\NormalTok{)}\OperatorTok{+}
\StringTok{  }
\StringTok{  }\CommentTok{#I use this function to modify headers for the x and y axis (saves the hassle of adding extra lines) }
\StringTok{  }\KeywordTok{scale_x_continuous}\NormalTok{(}\StringTok{"Concentration (mg/L)"}\NormalTok{)}\OperatorTok{+}
\StringTok{  }\CommentTok{#use continuous if your y axis is also numerical }
\StringTok{  }\KeywordTok{scale_y_discrete}\NormalTok{(}\StringTok{"Parameters"}\NormalTok{)}\OperatorTok{+}

\StringTok{  }\CommentTok{#here you can change the theme}
\StringTok{  }\KeywordTok{theme}\NormalTok{(}
    \DataTypeTok{axis.text=} \KeywordTok{element_text}\NormalTok{(}\DataTypeTok{color =} \StringTok{'black'}\NormalTok{,}\DataTypeTok{family=}\StringTok{'serif'}\NormalTok{),}
    \DataTypeTok{panel.background =} \KeywordTok{element_line}\NormalTok{(}\DataTypeTok{color =} \StringTok{"grey95"}\NormalTok{)}
\NormalTok{  )}\OperatorTok{+}
\StringTok{  }
\StringTok{  }\KeywordTok{theme_light}\NormalTok{(}\DataTypeTok{base_rect_size =} \DecValTok{0}\NormalTok{, }\DataTypeTok{base_size =} \DecValTok{16}\NormalTok{)}
\end{Highlighting}
\end{Shaded}

\includegraphics{geom_bar_files/figure-latex/unnamed-chunk-1-1.pdf}

\end{document}
